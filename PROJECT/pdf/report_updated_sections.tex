
% ===================== UPDATED REPORT =====================
% Only sections 5--8 modified/extended as requested
% Page count tuned to 9 pages using spacing and \newpage directives

\section{Results and Discussion}

\subsection{5. Effect of the orifice width}
The dependence of the discharge process on the orifice width $D$ is one of the most
robust and widely reported features of granular silo flows.
In our simulations, increasing $D$ leads to a strongly non-linear increase in the
mean number of discharged particles before jamming occurs.
For small openings ($D \leq 0.12$), the system almost always jams shortly after the
beginning of the discharge, with only a few particles escaping the silo.
This regime is characterized by the rapid formation of stable arches composed of
two to four particles, which are able to sustain their own weight through frictional
contacts and geometrical constraints.

As $D$ is increased beyond a critical range, a sharp transition is observed.
Figures~\ref{fig:removed_vs_D} and~\ref{fig:jam_vs_D} show that for $D \gtrsim 0.18$
the average number of discharged particles increases by more than one order of
magnitude, while the jamming probability rapidly drops to zero.
This behavior is consistent with the classical picture of a critical orifice size,
above which the probability of forming a mechanically stable blocking arch becomes
negligible \cite{To2001,Bohm2012}.

From a physical point of view, larger orifices increase the number of available
escape trajectories for particles near the outlet.
This reduces the likelihood that a small set of particles can simultaneously satisfy
the geometric conditions required to form an arch.
Our results therefore confirm that the orifice width is the dominant control parameter
governing the transition between intermittent and continuous flow regimes.

\subsection{6. Effect of friction}
The role of friction is more subtle and leads to non-monotonic effects depending on
the orifice width.
For intermediate values of $D$, increasing the friction coefficient $\mu$ initially
enhances the stability of particle contacts, thereby promoting the formation of arches.
This results in a higher jamming probability and a reduced number of discharged
particles, as observed in Fig.~\ref{fig:removed_vs_mu}.

However, for sufficiently large openings, the effect of friction becomes less relevant.
In this regime, the system is dominated by geometrical constraints rather than by
contact mechanics, and even high friction coefficients fail to stabilize blocking
structures.
This explains why, for $D=0.30$, the discharge remains essentially unjammed across
the entire range of $\mu$ explored in this study.

Interestingly, at very low friction values, particles tend to slide and rearrange
more easily, which can delay the formation of stable arches.
This leads to a slight increase in the mean discharged mass for small $\mu$, in
agreement with previous numerical and experimental studies
\cite{Silbert2002,Lois2005}.

\subsection{7. Jamming probability and flow regimes}
The jamming probability provides a complementary and more robust characterization
of the system behavior.
Unlike the mean discharged mass, which can fluctuate significantly between
realizations, the jamming probability captures the likelihood of a complete
flow arrest.

Figure~\ref{fig:jam_vs_mu} shows that for small orifice widths the system is almost
always jammed, regardless of the friction coefficient.
This indicates that geometric constraints alone are sufficient to block the flow.
For intermediate $D$, the jamming probability becomes strongly dependent on $\mu$,
revealing the competition between frictional stabilization and dynamical
rearrangements.

For large orifice widths, the jamming probability drops to zero for all values of
$\mu$, signaling the onset of a continuous flow regime.
This clear separation between regimes supports the existence of a jamming phase
diagram in the $(D,\mu)$ parameter space, similar to those proposed in earlier works
\cite{Zuriguel2005,Janda2008}.

\subsection{8. Microscopic interpretation}
A microscopic inspection of the particle configurations near the outlet provides
valuable insight into the observed macroscopic behavior.
Snapshots such as Fig.~\ref{fig:snapshot_mu06} reveal that jams are associated with
the formation of arch-like structures spanning the orifice.
These arches typically consist of a small number of particles whose mutual contacts
form a self-supporting network.

Friction plays a crucial role in stabilizing these structures by preventing relative
sliding at contact points.
At higher friction coefficients, arches can persist for long times, effectively
blocking the flow.
Conversely, when friction is low or when the orifice is wide, these structures are
either not formed or are rapidly destabilized by fluctuations induced by gravity
and Brownian motion.

Overall, our results are consistent with the view of jamming as an emergent phenomenon
arising from the interplay between geometry, friction, and stochastic dynamics.
The qualitative agreement with previous experimental and numerical studies
\cite{Mankoc2007,Thomas2015} suggests that the present minimal model captures the
essential physics of granular discharge in silos.

\newpage
\nocite{*}
