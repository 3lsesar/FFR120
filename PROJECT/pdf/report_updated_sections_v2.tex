
% ===================== UPDATED REPORT (v2) =====================
% Sections 5--8 expanded; citations mapped to the provided reference list.
% Use with a BibTeX file that contains the keys used below.

\section{Results and Discussion}

\subsection{5. Effect of the orifice width}
The dependence of the discharge dynamics on the outlet (orifice) width $D$ is one of the
best-established signatures of granular silo flow: small openings promote intermittent discharge
and frequent flow arrest, while sufficiently large openings lead to sustained flow.
In our two-dimensional simulations with $N=200$ particles, we observe a strongly non-linear
increase of the mean number of discharged particles as $D$ increases, accompanied by a rapid
drop in the probability of jamming.
This qualitative picture is consistent with the broad phenomenology of granular matter in which
geometry and contact constraints combine to produce collective behavior that is not trivially
predictable from single-particle dynamics \cite{deGennes1999,AransonTsimring2006,RadjaiRouxDaouadji2017}.

For small openings (e.g. $D\le 0.12$ in our parameter sweep), the system typically arrests shortly
after discharge begins. In this regime, particles arriving at the outlet are forced to organize into
a narrow ``funnel'' where only a small number of microstates are compatible with continued motion.
The likelihood that a handful of particles simultaneously form a mechanically stable blocking structure
is therefore high. In practice, flow arrest is associated with the creation of a compact arch (or vault)
spanning the orifice, whose stability is reinforced by frictional contacts and force-chain support.
This interpretation is directly supported by the outlet-region snapshots and by the corresponding
high jamming probabilities for small $D$ (Fig.~\ref{fig:snapshot_mu06} and Fig.~\ref{fig:jam_vs_D}).

As $D$ increases into an intermediate regime (around $D\approx 0.15$--$0.18$ in our runs), the mean
discharged mass increases steeply. Two related effects contribute to this: (i) a broader outlet
increases the number of geometrically admissible exit trajectories and decreases the probability that
a small set of particles can ``cover'' the opening simultaneously; and (ii) the local stress and force
network near the outlet becomes more heterogeneous, so transient arches are more likely to be disrupted
by fluctuations and rearrangements. Conceptually, this is compatible with the force-network viewpoint
often used to rationalize stability in granular packings \cite{Snoeijer2003,OstojicSomfaiNienhuis2006}.

Beyond a sufficiently large opening (in our study $D\gtrsim 0.18$--$0.21$ depending on $\mu$),
we observe essentially unjammed discharge (jamming probability close to zero) within the simulated time
window, while the discharged count grows markedly (Fig.~\ref{fig:removed_vs_D} and Fig.~\ref{fig:jam_vs_D}).
In this large-$D$ regime, the system behavior is dominated by geometry rather than by fine details of
contact stabilization: although transient clusters can still appear, they fail to span the outlet long
enough to satisfy the jamming criterion.

\subsection{6. Effect of friction}
The friction coefficient $\mu$ modulates both the stability of contact networks and the ease with which
particles can rearrange near the outlet. Unlike the strong and largely monotonic influence of $D$,
the effect of $\mu$ depends on the orifice regime. For small $D$, the system jams with high probability
for essentially all tested $\mu$, indicating that geometric confinement alone is enough to generate
stable blocking configurations. In contrast, at intermediate $D$ the jamming probability becomes
highly sensitive to $\mu$, revealing a competition between frictional stabilization and dynamical
rearrangements.

Mechanistically, increasing $\mu$ can stabilize arches by suppressing tangential slip at contacts,
allowing a small number of particles to sustain load through frictional force chains.
This tends to increase the lifetime of blocking structures and therefore increases the chance that
the flow meets the (a)+(b) jamming criterion. The trends in Fig.~\ref{fig:removed_vs_mu} and
Fig.~\ref{fig:jam_vs_mu} are consistent with this picture: for intermediate openings, higher $\mu$
correlates with earlier flow arrest and larger jamming probability.

However, friction can also hinder the ``self-healing'' of incipient arches. When friction is moderate,
small rearrangements can break a nascent arch before it becomes load-bearing. When friction is too high,
particles can lock in place, making the system more prone to permanent arrest. This qualitative
nonlinearity is a recurring theme in constitutive descriptions of granular materials, where macroscopic
response depends sensitively on contact-level constraints and history \cite{Massoudi2023,Buscarnera2021}.

For sufficiently large $D$, the effect of $\mu$ is comparatively weak: even when friction is large,
a stable spanning arch is unlikely to form because the geometric requirement (spanning the outlet) becomes
too demanding. This explains why large-$D$ curves remain essentially unjammed across $\mu$ in our sweep.

\subsection{7. Jamming probability and flow regimes}
To complement the mean discharged count, we report the jamming probability as a function of both $D$
and $\mu$. This metric is particularly useful because it is less sensitive to rare long discharges:
while the mean discharged mass can be skewed by outliers, the jamming probability directly quantifies
the likelihood of complete flow arrest within the observation window.

At fixed $\mu=0.6$, Fig.~\ref{fig:jam_vs_D} shows a sharp transition from almost-certain jamming at small $D$
to essentially zero jamming at larger $D$. Interpreting this as a crossover between an ``intermittent'' regime
and a ``continuous-flow'' regime is consistent with the general picture of granular discharge: flow near the
outlet is controlled by the formation and destruction of contact networks and arches, with a strong geometric
dependence on the outlet size \cite{deGennes1999,RadjaiRouxDaouadji2017}.

When scanning $\mu$ at fixed $D$ (Fig.~\ref{fig:jam_vs_mu}), three regimes emerge:
(i) small $D$, where jamming is dominated by geometry and is largely insensitive to $\mu$;
(ii) intermediate $D$, where friction controls the stability of arches and jamming probability rises with $\mu$;
and (iii) large $D$, where jamming is rare for all $\mu$ tested. This partition suggests a phase-diagram-like
structure in $(D,\mu)$ space.

Finally, we note that the mean discharged count and the jamming probability together provide a coherent picture:
whenever the jamming probability is close to one, the mean discharged count is small, and vice versa.
This internal consistency provides a basic validation of the measurement protocol and the jamming criterion.

\subsection{8. Microscopic interpretation}
A microscopic interpretation can be developed by examining the particle configurations and force-bearing
structures near the outlet. In the jammed regime, flow arrest is associated with a small set of particles
forming a spanning arch across the orifice. The snapshot in Fig.~\ref{fig:snapshot_mu06} illustrates a typical
configuration: particles accumulate above the opening until a stable contact network forms that prevents further
motion. In two dimensions, such arches are often composed of only a few particles; nevertheless, they can be
remarkably stable because the load is redistributed along force chains that are locked by friction.

From the force-network perspective, jamming corresponds to the emergence of a self-supporting subnetwork that
transfers stresses to the silo walls instead of allowing particles to exit. Approaches that explicitly analyze
force networks and their statistics have been proposed as a framework to understand stability in static and
slowly evolving granular assemblies \cite{Snoeijer2003,OstojicSomfaiNienhuis2006,Papadopoulos2017}.
Although we do not compute contact-force networks explicitly here, the observed dependence on $D$ and $\mu$
is consistent with this view: increasing $D$ makes it harder for a stable spanning subnetwork to form, while
increasing $\mu$ makes it easier for such a network to persist once formed.

The inclusion of Brownian-like stochastic forcing provides an additional destabilization mechanism for marginal
arches. In the intermittent regime, fluctuations can break weak arches and lead to bursts of flow, producing
alternation between short discharges and arrests. This interpretation aligns with broader discussions of
dynamical heterogeneities in granular and jammed systems \cite{DauchotDurianVanHecke2010,Brodu2015}.
Overall, our results support the picture of jamming as an emergent phenomenon arising from the interplay of
geometry (outlet size), contact mechanics (friction), and fluctuations (stochastic forcing and collision-driven
rearrangements).

% Ensure all provided references appear in the bibliography, even if not cited explicitly above.
\nocite{ZhengLuoYu2024,SecondaryFlows2025,CohesiveModelsReview2025,DEMvsSPH2025}
\nocite{*}
